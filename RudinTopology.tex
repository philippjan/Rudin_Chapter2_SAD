\documentclass[10pt]{article}
\usepackage[utf8]{inputenc}
\usepackage[T1]{fontenc}
\usepackage{lmodern}
\usepackage[english]{babel}
\usepackage{amsmath,amssymb,amsfonts,amsthm,mathtools,sansmath}
\usepackage{cleveref}
\usepackage{framed}

\newenvironment{forthel}{\begin{leftbar}}{\end{leftbar}}

%Commands
\newcommand{\model}{\mathfrak{M}}
\newcommand{\claim}{\textsf{Behauptung}:\hspace{0,2cm}}
\newcommand{\N}{\mathbb{N}}
\newcommand{\Fin}{\mathsf{Fin}}
\newcommand{\Dom}{\mathsf{Dom}}
\makeatletter
\def\fall#1{\forall #1\@ifnextchar\bgroup{\,\fall}{:\,}}
\makeatother

%Umgebungen
\newtheorem{theorem}{Theorem}[section] 
\theoremstyle{definition}
\newtheorem{definition}[theorem]{Definition}
\theoremstyle{plain}
\newtheorem{corollary}[theorem]{Korollar}
\newtheorem{signature}[theorem]{Signature}
\newtheorem{lemma}[theorem]{Lemma}
\newtheorem{proposition}[theorem]{Proposition}
\newtheorem{axiom}[theorem]{Axiom}
\theoremstyle{remark}
\newtheorem{remark}[theorem]{Remark}
\newtheorem{example}[theorem]{Example}

\title{Basic Topology: Finite and Infinite Sets}
\author{Philipp Stassen}
\begin{document}
\maketitle
In this paper we formalize the first three pages (page 24, 25, 26) of the Chapter ''Basic Topology'' of Walter Rudin's ''\emph{principles of mathematical analysis}''. 
This formal text shall then be verified by SAD 3.0, which itself utilizes the eprover to carry out proof tasks.

We would like to stick with the formulations in the book. This is however not always possible as we might adjust certain phrases to assist the automated deduction. It is furthermore necessary to specify certain details that are usually left implicit.

We need to start by an appropriate axiomatization of the natural numbers.
\begin{forthel}
[set/-s] [element/-s] [subset/-s] [sequence/-s] [number/-s]
	\begin{signature}[NaturalNumbers]
		A natural number is a notion.
	\end{signature}
	Let $x,y,z,m,n$ denote natural numbers.
	\begin{signature}[Addition]
		$x + y$ is a natural number.
	\end{signature}
	\begin{signature}
		$0$ is a natural number.
	\end{signature}
	\begin{signature}
		$1$ is a natural number.
	\end{signature}
	\begin{axiom}
		$x + 0 = x$.
	\end{axiom}
	\begin{axiom}[AssAdd]
	 	$(x+y)+z=x+(y+z)$.
	\end{axiom}
	\begin{axiom}[ComAdd]
		$x+y=y+x$.
	\end{axiom}
	Let $x \neq 0$ stand for not $x=0$.

	Let x is nonzero stand for not $x=0$.
	\begin{axiom}
		$1 \neq 0$.
	\end{axiom}
	\begin{axiom}[NatPredecessor]
		For every nonzero $m$ there exists $n$ such that $m = n+1$.
	\end{axiom}
	\begin{axiom}[NonZeroAddition]
		Let $n$ be a natural number. For all nonzero natural numbers $m$ $n+m \neq 0$.
	\end{axiom}
	\begin{axiom}
		For all natural numbers $n,m$ if $n+1=m+1$ then $n=m$.
	\end{axiom}
	\begin{definition}[Less]
		$x<y$ iff there exists a nonzero natural number $z$ such that $x+z=y$.
	\end{definition}
\end{forthel}
This axiomatization allows us to prove the following lemmas.
\begin{forthel}
	\begin{lemma}[SmallestZero]
		For all nonzero natural numbers $x$ $0<x$.
	\end{lemma}
	\begin{lemma}[Successorgreater]
		For all natural numbers $x$ $x<x+1$.
	\end{lemma}
	\begin{lemma}[Monotonicity]
		Let $n,m,x$ be natural numbers. If $n<m$ then $n+x < m+x$.
	\end{lemma}
	\begin{proof}
			Assume $n < m$.
			Take a nonzero natural number $z$ such that $n+z=m$.
			\begin{align*}
				x + (n+m) &.= (x+n)+m \tag*{AssAdd, ComAdd} \\
					  &.= m+x.
			\end{align*}
			Then $z$ is a nonzero natural number such that $(n+x)+z=m+x$.
			Then $n+x < m+x$.
	\end{proof}
	\begin{lemma}[MonotonicityOne]
		Let $n,m$ be natural numbers. If $n<m$ then $n+1<m+1$.
	\end{lemma}
	\begin{lemma}[AssLess]
		If $x<y$ and $y<z$ then $x<z$.
	\end{lemma}
	\begin{proof}
		Assume $x<y$. Take a nonzero natural number $n$ such that $x+n=y$. Assume $y<z$. Take a nonzero natural number $m$ such that $y+m=z$.
		\begin{align*}
			x+(n+m)&.=(x+n)+m \tag*{AssAdd}\\
			       &.=y+m\\
			       &.=z.
		\end{align*}
		Then $n+m$ is a nonzero natural number such that $x+(n+m)=z$ (by NonZeroAddition). Then $x<z$ (by Less).
	\end{proof}
\end{forthel}
We would like to include an induction scheme to prove theorems by induction on the natural numbers. Therefore, we make use of the built in $\prec$ relation. By ''embedding '' the $<$ relation into the $\prec$ relation we can use recursion and induction on $<$ as for $\prec$. At this point we automatically enrich our theory by an axiom scheme that presupposes the well-foundedness of $<$.
\begin{forthel}
	\begin{signature}[InbuiltFortheInductionRelation] 
	$m \prec n$ is an atom.
	\end{signature}
	\begin{axiom}[EmbeddingLessIntoInductionRelation] 
	$m < n \Rightarrow m \prec n$.
	\end{axiom}
\end{forthel}
Now we can prove the ''trichotomy'' of the $<$ - relation by induction. 
\begin{forthel}
	\begin{lemma}
		For all natural numbers $x$ for all natural numbers $y$ $x=y$ or $x<y$ or $y<x$.
	\end{lemma}
	\begin{proof}[Proof by induction on x]
		Let $x$ be a natural number.\\
		\textbf{Case} $x=0$. Let $y$ be a natural number. \\
		\textbf{Case} $y=0$. \\
			$x .= 0 .= y$. end. \\
		\textbf{Case} $y$ is nonzero. \\
		$x<y$ (by SmallestZero) end. \\
		end. \\
		\textbf{Case} $x \neq 0$. \\
		For all natural numbers $a$ $x=a$ or $x<a$ or $a<x$.
		\begin{proof}[Proof by induction on a]
			Let $a$ be a natural number. \\
			\textbf{Case} $a=0$. \\
			$a<x$ (by SmallestZero). end. \\
			\textbf{Case} $a\neq 0$. \\
			Take $m$ such that $m+1=x$ (by NatPredecessor). Then $m<x$ (by Successorgreater). \\
			Take $n$ such that $n+1=a$ (by NatPredecessor). Then $n<a$ (by Successorgreater). \\
			\textbf{Case} $m=n$. \\
			\begin{align*}
				x &.= m+1 \\
				  &.= n+1 \\
				  &.= a.
			\end{align*}
			end. \\
			\textbf{Case} $m<n$.\\
			Then $m+1 < n +1$ (by MonotonicityOne). Then $x<a$. end. \\
			\textbf{Case} $n<m$. \\
			Then $n+1 < m+1$ (by MonotonicityOne). Then $a<x$. end. \\
			end.
		\end{proof}
		end.
	\end{proof}
\end{forthel}
Now that we defined the natural numbers we can define a notion of finiteness that is similar to the one that Walter Rudin defines at the beginning of chapter two.
\begin{forthel}
	\begin{definition}[Subset]
		Let $T$ be a set. A subset of $T$ is a set $S$ such that every element of $S$ is an element of $T$.
	\end{definition}
	\begin{definition}[Complement]
		Let $M,N$ be sets. $M\setminus N = \{x \text{ in }M | x \text{ is not an element of }N\}$.
	\end{definition}
	Let $S\subset T$ stand for $S$ is a subset of $T$.
	\begin{definition}
		Let $T$ be a set. A proper subset of $T$ is a set $S$ such that $S\subset T$ and there exists and element of $T$ that is not an element of $S$.
	\end{definition}
	\begin{definition}
		Let $n$ be a natural number. $\Fin(n) = \{\text{natural number }m | m < n\}$.
	\end{definition}
	\begin{definition}
		Let $S$ be a set. $S$ is empty iff $S$ has no elements.
	\end{definition}
	Let $S$ is nonempty stand for S is not empty.
	\begin{definition}[Union]
		Let $S$ be a set such that every element of $S$ is a set. \\
		$\bigcup S = \{b |\text{ there exists an element }T\text{ of }S\text{ such that }b\text{ is an element of }T \}$.
	\end{definition}
	\begin{definition}[Intersection]
		Let $S$ be a set such that every element of $S$ is a set. \\
		$\bigcup S = \{b |\text{ for all elements }T\text{ of }S\text{ such that }b\text{ is an element of }T \}$.
	\end{definition}
	\begin{definition}[surjective]
		Let $T$ be a set. A Surjection on $T$ is a function $f$ such that for every element $t$ of $T$ there exists an element $s$ of $Dom(f)$ such that $f[s]=t$.
	\end{definition}
	Let $f$ is surjective on T stand for $f$ is a Surjection on $T$.
	\begin{definition}[injective]
		Let $f$ be a function. $f$ is injective iff for all elements $x,y$ of $Dom(f)$ we have $(x \neq y \Rightarrow f[x] \neq f[y])$.
	\end{definition}
	\begin{definition}
		Let $S,T$ be sets. \\
		$S \twoheadrightarrow T$ iff there exists a function $f$ such that $f$ is surjective on $T$ and $Dom(f)=S$
	\end{definition}
	\begin{definition}
		Let $S$ be a set. Let $n$ be a natural number. \\
		$S$ has cardinality $n$ iff $\Fin(n) \twoheadrightarrow S$ and for all natural numbers $m$ if $m<n$ then not $(\Fin(m) \twoheadrightarrow S$.
	\end{definition}
	\begin{definition}
		Let $S$ be a set. Let $T$ be a set. $S$ and $T$ are equipotent iff there exists a natural number $n$ such that $S$ has cardinality $n$ and $T$ has cardinality $n$.
	\end{definition}
	Let $S\sim T$ stand for $S$ and $T$ are equipotent.
	\begin{definition}
		Let $S,T$ be sets. \\
		$S \leftrightarrow T$ iff there exists a function $f$ such that $f$ is surjective on $T$ and $Dom(f) = S$ and $f$ is injective.
	\end{definition}
	\begin{definition}
		Let $S$ be a set. $S$ is finite iff there exists a natural number $n$ such that $S$ has cardinality $n$.
	\end{definition}
\end{forthel}
It is unclear whether this notion of finiteness is sensible. It might be to complicated for the prover. Simple properties as the fact that $\sim$ is an equivalence relation can however be verified.
\begin{forthel}
	\begin{lemma}
		Let $S$ be a finite set. Then $S \sim S$.
	\end{lemma}
	\begin{lemma}
		Let $S,T$ be finite sets. If $S\sim T$ then $T\sim S$.
	\end{lemma}
	\begin{lemma}
		Let $S,T,U$ be finite sets. If $S \sim T$ and $T \sim U$ then $S \sim U$.
	\end{lemma}
\end{forthel}
Another important notion is the countability of sets.
\begin{forthel}
	\begin{definition}
		$\N = \{ a | \text{ a is a natural number }\}$.
	\end{definition}
	\begin{definition}[List]
		Let $S$ be a set. A list of $S$ is a function $f$ such that $Dom(f) = \N$ and for every element $x$ of $S$ there exists a natural number $n$ such that $f[n] = x$.
	\end{definition}
	\begin{definition}[Ctbl]
		Let $S$ be a set. \\
		$S$ is countable iff there exists a function $f$ such that $f$ is a list of $S$.
	\end{definition}
	\begin{definition}
		Let $M$ be a set. Let $N$ be a subset of $M$. Let $f$ be a list of $M$. Let $x$ be an element of $N$. $x$ is smallest with respect to $f$ and $M$ in $N$ iff there exists a natural number $n$ such that $f[n]=x$ and for every natural number $m$ such that $m < n$ $f[m]$ is not an element of $N$.
	\end{definition}
	\begin{definition}
		Let $f$ be a function such that $Dom(f)= \N$. Let $n$ be a natural number. $f << n = \{f[m]| m\text{ is a natural number and } m<n \}$.
	\end{definition}
\end{forthel}
This notion of countability allows us to prove that every subset of a countable set is again countable.
\begin{forthel}
	\begin{proposition}
	Le $M$ be a set. If $M$ is countable then every subset of $M$ that has an element is countable.
	\end{proposition}
	\begin{proof}
		Assume $M$ is countable. Take a list $f$ of $M$. Let $N$ be a subset of $M$ that has an element. Define \\
		$g[n] = $ \\
		\hspace*{1em}\textbf{Case} $((M\setminus (f << n))$ has an element) -> Choose an element a of $(M\setminus (f << n))$ that is smallest with respect to f and M in $(M\setminus(f << n))$ in a, \\
		\hspace*{1em}\textbf{Case} $((M\setminus(f<<n))$ has no element) -> Choose an element a of N in a \\
		for $n$ in $\N$.
		\begin{proof}
			Choices.
		\end{proof}
	\end{proof}
\end{forthel}
\section{Concluding Remarks} SAD 3.0 seems to be capable to deal with basic notions like natural numbers, functions and sets and more complex derived notions like cardinalities or lists. However, complex notions like the finiteness of sets will be difficult to use and require a lot of preparations. Therefore, it seems more reasonable to use a primitive notion of finiteness rather than a derived one. 

I think that the possibility to define functions by recursion and verify proofs by induction should suffice to work with finiteness as defined in this paper. It might run up against hardware limitations because of complexity issues though.
\end{document}
